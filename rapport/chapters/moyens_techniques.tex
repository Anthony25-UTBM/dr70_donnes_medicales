    \section{Moyens techniques}

        \subsection{Stockage des données}
        \subsubsection{Agrément}

\paragraph{}
Pour pouvoir stocker des données médicales, un hébergeur doit obtenir un
agrément de la part de l'ASIP Santé (Agence des Systèmes d'Information Partagés
de Santé), délivré par le ministre en charge de la santé pour une durée de
trois ans. La seule exception à la nécessité de posséder cet agrément est dans
le cas de la conservation de données hospitalières, qui peuvent être conservés
durant 20 ans. Dans le cas d'une première demande, la société doit remplir un
formulaire qui sera évalué par l'ASIP qui va poursuivre ou non les démarches.

\paragraph{}
Selon
\href{http://www.legifrance.gouv.fr/affichCodeArticle.do;jsessionid=B4723D30F71044611FD12BFBB7948576.tpdila07v_3?idArticle=LEGIARTI000023676881&cidTexte=LEGITEXT000006072665}{Legifrance}\footnote{\url{http://www.legifrance.gouv.fr/affichCodeArticle.do;jsessionid=B4723D30F71044611FD12BFBB7948576.tpdila07v_3?idArticle=LEGIARTI000023676881&cidTexte=LEGITEXT000006072665}},
les conditions nécessaires pour se voir accorder
l'agrément sont les suivantes :
\begin{enumerate}
    \item La structure qui accueillera les données médicales est jugée comme
        fiable en terme de conservation et sécurité. L'hébergeur doit avoir
        recours à des personnels qualifiés dans ces deux domaines, et doit
        avoir en place une organisation, des procédures de contrôles et des
        solutions techniques garantissant la protection, la sauvegarde et la
        restitution des données confiées.
    \item Une politique de confidentialité et de sécurité doit être définie et
        mise en place, de façon à garantir le respect des exigences de
        confidentialité et de secret\footnote{Prévues par les articles
        L.\,1110--4 et L.\,1111--7} ainsi que la protection contre les accès
        non autorisés.
    \item L'activité d'hébergement des données médicales doit être séparée du
        reste de l'activité de l'entreprise, ainsi que la gestion des stocks et
        des flux de données associés.
    \item Avoir des dispositifs d'informations en place sur l'activité
        d'hébergement pour les patients.
    \item Renseigner les personnes en charge de l'hébergement, qui doit
        comprendre obligatoirement un médecin.
\end{enumerate}

\paragraph{}
De façon à vérifier que ces conditions sont validées, un audit est demandé par
l'ASIP\@. Lors du renouvellement, celui ci est externe et à la charge de
l'hébergeur, autrement il est effectué par l'ASIP dans cadre de la première
demande.

        \subsection{Accès aux données}
% carte vitale
% qui peut stocker les données ?
