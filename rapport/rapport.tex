\documentclass{article}

\usepackage[utf8]{inputenc}
\usepackage[T1]{fontenc}
\usepackage[francais]{babel}
\usepackage{multirow}
\usepackage{array}
\usepackage{color}
\usepackage{stmaryrd}
\usepackage{fancyhdr}
\usepackage{afterpage}
\usepackage{fullpage}
\usepackage{geometry}
\usepackage{setspace}
\usepackage{enumitem}
\usepackage{hyperref}


% Enlève les contours des liens
\hypersetup{
    linkbordercolor={1 1 1},
    citebordercolor={1 1 1},
    urlbordercolor={1 1 1},
    colorlinks=true,
    linkcolor=black,
    urlcolor=blue
}
\PassOptionsToPackage{hyphens}{url}\usepackage{hyperref}


% Redéfinis les marges des tableaux
\let\oldtabular=\tabular
\def\tabular{\small\oldtabular}
\renewcommand{\arraystretch}{1.5}


\title{\textbf{DR70 - Le dossier médical patient}}

\author{Benoit HOUDAYER \\ \href{mailto:benoit.houdayer@utbm.fr}{benoit.houdayer@utbm.fr}
\and Anthony RUHIER \\ \href{mailto:anthony.ruhier@utbm.fr}{anthony.ruhier@utbm.fr}}

\date{04 janvier 2016}

\pagestyle{fancy}
\setlength{\headheight}{12pt}
\fancyhf{}
\fancyhead[L]{Benoit HOUDAYER, Anthony RUHIER}
\fancyhead[R]{DR70 - Le dossier médical patient}
\geometry{headsep=5ex}


\begin{document}
    \maketitle
    \thispagestyle{empty}
    \vspace{4em}
    \tableofcontents

    \afterpage{\cfoot{\thepage}}
    \newpage

%%%% Includes des chapitres :
%%%%%%%%%%%%%%%%%%%%%%%%%%%%%%
\setcounter{page}{1}
    \section*{Introduction}
\addcontentsline{toc}{section}{Introduction}

\paragraph{}
Le dossier médical est un ensemble de documents destiné à suivre l'état de
santé d'un patient. Les données nécessaires à la constitution d'un tel document
sont détaillées dans le Code de la Santé Publique, et chaque établissement est
tenu de s'assurer que le dossier d'un patient est correctement constitué.

\paragraph{}
Aujourd'hui, la plupart des établissement optent pour des solutions de gestion
des dossiers informatisées; en conséquence, des problématiques nouvelles sont
soulevées à propos de ces données.

\paragraph{}
La vie privée des patients est un des enjeux les plus importants, les données à
caractère médical sont très étroitement surveillées par la Commission Nationale
de l'Informatique et des Libertés. Dans le même temps, les possibilités
offertes par le support numérique ouvrent la voie à la modernisation et à une
efficacité accrue du système de suivi des patients.

\paragraph{}
Pour permettre la modernisation du domaine de la santé, la loi évolue et
s'adapte pour faciliter l'accès à l'information, tout en prenant les
précautions nécessaires pour préserver la vie privée des patients.

\paragraph{}
Le présent document vise à présenter le dossier médical patient et les
contraintes de confidentialité qui y sont liées, puis de s'intéresser aux
moyens d'informatisation des dossiers et aux mesures utilisés pour préserver
les données des patients.

    \section{Le dossier médical}


\paragraph{}
Le dossier médical patient désigne l'ensemble des informations recueillies par
les professionnels du domaine médical à propos de l'état de santé d'un patient.
Ces données, stockées sous forme de numérique ou sur papier sont indispensables
pour assurer le suivi d'un patient. Bien que dans la plupart des établissements
médicaux, les dossiers sont stockés sous forme numérique le partage des données
n'est pas systématique avec d'autres établissements.

\paragraph{}
Bien au contraire, la CNIL veille au respect de la vie privée des patients,
notamment en définissant des normes de sécurité et des règles d'accès très
strictes. Pour autant, l'accès au données est quasi-systématiquement possible
pour le patient, et à d'autres personnes dans des cas restreints, très encadrés
par la loi.


        \subsection{Le contenu du dossier médical}

\paragraph{}
Les informations contenues dans le dossier médical sont initialement prévues
pour documenter l'état de santé d'un patient et suivre ses traitements lors de
son admission dans un établissement de santé.  Un professionnel de la santé
travaillant avec le patient a accès à ce dossier pour déposer les données
formalisées.

\paragraph{}
Le terme ``formalisé'' n'est pas défini, mais la loi du 4 mars 2002 prévoit que
``les informations formalisées accessibles au patient doivent s'entendre comme
présentant un certain degré d'élaboration et de validation''. Par ailleurs,
d'après une décision de la Cour d'Appel de Paris, les lames d'examen biologique
ne sont pas des informations
formalisées\footnote{\url{http://documentation.fhp.fr/documents/caparis_20080213.pdf}}

\paragraph{}
Les informations que le dossier médical est susceptible de contenir sont et
réparties en trois catégories~:

\begin{itemize}
    \item Les informations recueillies par l'établissement lors de
        consultations.

        Ces informations comprennent le motif d'admission dans l'établissement, les
        soins administrés, les résultats d'examens médicaux.

        Cette catégorie comprend également les correspondances entre les
        professionnels de la santé.

    \item Les informations établies en fin de séjour d'un patient dans un
        établissement.

        Cette catégorie comprend le compte rendu d'hospitalisation et la lettre
        rédigée à la sortie du patient, les prescriptions et ordonnances.

    \item Les informations recueillies auprès de personnes tierces, ou à propos
        de personnes tierces.

        De telles données peuvent être pertinentes dans un dossier médicales, mais
        doivent être consignées à part préserver la vie privée des tiers en
        question.
\end{itemize}

\paragraph{}
Le patient possède un droit d'accès et de rectification au dossier, comme prévu
par la loi informatique et libertés, à l'exception du contenu de la troisième
catégorie, susceptible de contenir des informations concernant une personne
tierce. Un exemple des données susceptibles de figurer dans la troisième
catégorie est l'analyse par un psychologue du comportement des proches d'un
patient~: les informations sont pertinentes dans le cadre du traitement du
patient, mais la vie privée des proches est une raison suffisante pour ne pas
divulguer l'information au patient.


        \subsection{L'accès aux données}

\paragraph{}
Lorsqu'un patient demande l'accès à son dossier, l'établissement médical est dans l'obligation
de lui fournir l'intégralité des données des deux premières catégories.

\paragraph{}
Dans certains cas, il est possible que l'accès au dossier soit proposé au patient sous conditions~:

\begin{itemize}
    \item Si le patient est hospitalisé en psychiatrie, la consultation du
        dossier peut être refusée, ou doit être faite en présence d'un médecin.

    \item Pour les patients mineurs, l'accès au dossier médical peut être donné
        aux parents, mais le patient mineur peut demander que l'accès aux
        informations se fasse par l'intermédiaire d'un médecin, et dans ce cas,
        s'opposer à ce que le médecin communique les informations.

    \item Dans le cas d'une personne décédée, les successeurs peuvent accéder
        au dossier médical complet ou en partie. Pour obtenir des informations,
        un successeur doit présenter un motif à la consultation du dossier. Les
        motifs valides de consultation sont~:
            \subitem{\textbullet\,} Prendre connaissance de la cause du décès.
            \subitem{\textbullet\,} Défendre la mémoire du défunt.
            \subitem{\textbullet\,} Faire valoir ses droit.
\end{itemize}

\paragraph{}
Le troisième motif résulte d'une décision du Conseil d'État par rapport à un
contentieux\footnote{\url{http://www.legifrance.gouv.fr/affichJuriAdmin.do?oldAction=rechJuriAdmin&idTexte=CETATEXT000008133767&fastReqId=583477576&fastPos=1}}~:
l'établissement refuse de communiquer le dossier médical d'une personne
décédée, sous prétexte que l'ayant droit dont émane la demande est en litige
avec un autre ayant droit, et que les informations son susceptibles d'être
utilisées dans le cadre du litige.  Le refus de communication du dossier dans
ce cas est alors jugé illégal.

Dans certains cas, les établissement échouent à fournir les dossier médicaux
complets ou bien dans les délais impartis par la loi. Par exemple, une décision
de la Cour d'Appel de Marseille établit que la communication partielle du
dossier médical d'un patient décédé, provoquant la confusion sur la cause du
décès constitue un préjudice moral à l'encontre des ayant droit du
patient.\footnote{\url{http://documentation.fhp.fr/documents/caamarseille_20080313.pdf}}

    \section{Moyens techniques}

    \subsection{La carte vitale}

    La carte Vitale a été mise en service dès 1998, elle permet d'identifier de manière unique un assuré social, et est un composant du
    système d'information SESAM-Vitale. En plus de l'identifiant unique, la carte contient
    les informations suivantes :

    \begin{itemize}
        \item l'identité du titulaire et de ses ayant-droits de moins de 16 ans
        \item le numéro de sécurité sociale du titulaire
        \item le régime d'assurance maladie et l'organisme de rattachement
        \item les droits à la couverture maladie universelle complémentaire (CMUC)
        \item les droits à l'exonération du ticket modérateur (frais médicaux normalement à la charge du patient)
    \end{itemize}

        Une nouvelle version, appelée Vitale 2 
    a quant à elle été introduite en 2007 pour remplacer progressivement les cartes existantes.
    À la différence de la première version, une photo de l'assuré est présente sur la face et des 
    informations supplémentaires sont présentes sur sa mémoire.

    \begin{itemize}
        \item le médecin traitant
        \item la mutuelle
        \item la personne à prévenir en cas d'accident
        \item les informations sur l'accès aux soins dans l'UE
        \item la gestion des accidents du travail
        \item la carte de donneur d'organe du titulaire
    \end{itemize}

    Techniquement, La carte vitale est équipée d'un microprocesseur lui permettant de chiffrer et
    déchiffrer ses données de manière autonome. En plus des mesures de chiffrement le protocole
    permettant de questionner la carte est maintenu secret. Ainsi, pour pouvoir lire ou écrire une carte vitale, 
    il est nécessaire d'utiliser une bibliothèque propriétaire, détenue par la caisse d'assurance maladie.

        \subsection{Stockage des données}
        \subsubsection{Agrément}

\paragraph{}
Pour pouvoir stocker des données médicales, un hébergeur doit obtenir un
agrément de la part de l'ASIP Santé (Agence des Systèmes d'Information Partagés
de Santé), délivré par le ministre en charge de la santé pour une durée de
trois ans. La seule exception à la nécessité de posséder cet agrément est dans
le cas de la conservation de données hospitalières, qui peuvent être conservés
durant 20 ans. Dans le cas d'une première demande, la société doit remplir un
formulaire qui sera évalué par l'ASIP qui va poursuivre ou non les démarches.

\paragraph{}
Selon
le Code de la Santé Publique \footnote{\url{http://www.legifrance.gouv.fr/affichCodeArticle.do;jsessionid=B4723D30F71044611FD12BFBB7948576.tpdila07v_3?idArticle=LEGIARTI000023676881&cidTexte=LEGITEXT000006072665}},
les conditions nécessaires pour se voir accorder
l'agrément sont les suivantes :
\begin{enumerate}
    \item La structure qui accueillera les données médicales est jugée comme
        fiable en terme de conservation et sécurité. L'hébergeur doit avoir
        recours à des personnels qualifiés dans ces deux domaines, et doit
        avoir en place une organisation, des procédures de contrôles et des
        solutions techniques garantissant la protection, la sauvegarde et la
        restitution des données confiées.
    \item Une politique de confidentialité et de sécurité doit être définie et
        mise en place, de façon à garantir le respect des exigences de
        confidentialité et de secret\footnote{Prévues par les articles
        L.\,1110--4 et L.\,1111--7} ainsi que la protection contre les accès
        non autorisés.
    \item L'activité d'hébergement des données médicales doit être séparée du
        reste de l'activité de l'entreprise, ainsi que la gestion des stocks et
        des flux de données associés.
    \item Avoir des dispositifs d'informations en place sur l'activité
        d'hébergement pour les patients.
    \item Renseigner les personnes en charge de l'hébergement, qui doit
        comprendre obligatoirement un médecin.
\end{enumerate}

\paragraph{}
De façon à vérifier que ces conditions sont validées, un audit est demandé par
l'ASIP\@. Lors du renouvellement, celui ci est externe et à la charge de
l'hébergeur, autrement il est effectué par l'ASIP dans cadre de la première
demande.

        \subsection{Accès aux données}
% carte vitale
% qui peut stocker les données ?

    \section{Conclusion}

\begin{frame}
\frametitle{Conclusion}
\begin{itemize}
    \itemsep2em
    \item Encadrement des hébergeurs rassurant
\end{itemize}
\end{frame}


\end{document}
