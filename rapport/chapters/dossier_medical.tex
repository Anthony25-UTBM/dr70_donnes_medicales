    \section{Le dossier médical}

\subsection{Le contenu du dossier médical}

Les informations contenues dans le dossier médical sont listées explicitement dans la loi et réparties en trois catégories :

\begin{itemize}

\item Les informations recueillies par l'établissement lors de consultations \newline{}
Ces informations comprennent le motif d'admission dans l'établissement, les soins administrés, 
les résultats d'examens médicaux, accompagnés des interprétations des professionnels.

Cette catégorie comprend également les correspondances entre les professionnels de la santé.

\item Les informations établies en fin de séjour d'un patient dans un établissement \newline{}
Cette catégorie comprend le compte rendu d'hospitalisation et la lettre rédigée à la sortie
du patient, les prescriptions et ordonnances.

\item Les informations recueillies auprès de personnes tierces, ou à propos de personnes tierces. \newline{}
De telles données peuvent être pertinentes dans un dossier médicales, mais doivent être
consignées à part préserver la vie privée des tiers en question.

\end{itemize}

\subsection{L'accès aux données}

Les patients peuvent demander à tout moment d'obtenir une copie de leur dossier médical. Lorsqu'un patient demande l'accès à son dossier, l'établissement médical est dans l'obligation
de lui fournir l'intégralité des données des deux premières catégories.

Dans certains cas, il est possible que l'accès au dossier soit proposé au patient sous conditions :


\begin{itemize}

    \item Si le patient est hospitalisé en psychiatrie, la consultation du dossier peut être
refusée, ou doit être faite en présence d'un médecin.

\item Pour les patients mineurs, l'accès au dossier médical peut être donné aux parents,
mais le patient mineur peut demander que l'accès aux informations se fasse par
l'intermédiaire d'un médecin, et dans ce cas, s'opposer à ce que le médecin communique les informations.

\item Dans le cas d'une personne décédée, les successeurs peuvent accéder au dossier médical
en partie seulement. Pour obtenir des informations, un successeur doit présenter un motif
à la consultation du dossier. Les motifs valides de consultation sont :

    \subitem prendre connaissance de la cause du décès.
    \subitem défendre la mémoire du défunt.
    \subitem faire valoir ses droit.

\end{itemize}

\subsection{Modes de consultation}

Pour l'instant, seuls deux modes de consultation sont possibles. Le patient peut décider de
consulter son dossier médical dans l'établissement médical le possédant, ou bien demander 
une copie physique envoyée directement à son domicile.


