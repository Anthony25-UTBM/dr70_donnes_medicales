    \section*{Introduction}
\addcontentsline{toc}{section}{Introduction}

\paragraph{}
Le dossier médical est un ensemble de documents destiné à suivre l'état de
santé d'un patient. Les données nécessaires à la constitution d'un tel document
sont détaillées dans le Code de la Santé Publique, et chaque établissement est
tenu de s'assurer que le dossier d'un patient est correctement constitué.

\paragraph{}
Aujourd'hui, la plupart des établissements optent pour des solutions de gestion
des dossiers informatisés~; en conséquence, des problématiques nouvelles sont
soulevées à propos de ces données.

\paragraph{}
La vie privée des patients est un des enjeux les plus importants, les données à
caractère médical sont très étroitement surveillées par la Commission Nationale
de l'Informatique et des Libertés (CNIL). Dans le même temps, les possibilités
offertes par le support numérique ouvrent la voie à la modernisation et à une
efficacité accrue du système de suivi des patients.

\paragraph{}
Pour permettre la modernisation du domaine de la santé, la loi évolue et
s'adapte pour faciliter l'accès à l'information, tout en prenant les
précautions nécessaires pour préserver la vie privée des patients.

\paragraph{}
Le présent document vise à présenter le dossier médical patient et les
contraintes de confidentialité qui y sont liées, puis de s'intéresser aux
moyens d'informatisation des dossiers et aux mesures utilisées pour préserver
les données des patients.
