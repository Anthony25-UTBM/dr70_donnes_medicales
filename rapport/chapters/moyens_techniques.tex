    \section{Moyens techniques}

\paragraph{}
Depuis 2006, des décrets précis ont été rédigés de façon à éviter toute fuite
de données médicales personnelles stockées sur support informatique, en se
concentrant sur la sécurité et les règles de confidentialité qu'un hébergeur
doit adopter. La propriété d'un agrément est rendue obligatoire pour stocker ce
type de données.

        \subsection{Agrément}


\paragraph{}
Pour pouvoir stocker des données médicales, un hébergeur doit obtenir un
agrément de la part de l'ASIP Santé (Agence des Systèmes d'Information Partagés
de Santé), délivré par le ministre en charge de la santé pour une durée de
trois ans. La seule exception à la nécessité de posséder cet agrément est dans
le cas de la conservation de données hospitalières, qui peuvent être conservés
durant 20 ans. Dans le cas d'une première demande, la société doit remplir un
formulaire qui sera évalué par l'ASIP qui va poursuivre ou non les démarches.

            \subsubsection{Stockage des données}

\paragraph{}
Selon le Code de la Santé
Publique\footnote{\url{http://www.legifrance.gouv.fr/affichCodeArticle.do;jsessionid=B4723D30F71044611FD12BFBB7948576.tpdila07v_3?idArticle=LEGIARTI000023676881&cidTexte=LEGITEXT000006072665}},
les conditions nécessaires pour se voir accorder l'agrément sont les
suivantes~:
\begin{enumerate}
    \item La structure qui accueillera les données médicales est jugée comme
        fiable en termes de conservation et sécurité. L'hébergeur doit avoir
        recours à des personnels qualifiés dans ces deux domaines, et doit
        avoir en place une organisation, des procédures de contrôles et des
        solutions techniques garantissant la protection, la sauvegarde et la
        restitution des données confiées.
    \item Une politique de confidentialité et de sécurité doit être définie et
        mise en place, de façon à garantir le respect des exigences de
        confidentialité et de secret\footnote{Prévues par les articles
        L.\,1110--4 et L.\,1111--7} ainsi que la protection contre les accès
        non autorisés.
    \item L'activité d'hébergement des données médicales doit être séparée du
        reste de l'activité de l'entreprise, ainsi que la gestion des stocks et
        des flux de données associés.
    \item Avoir des dispositifs d'informations en place sur l'activité
        d'hébergement pour les patients.
    \item Renseigner les personnes en charge de l'hébergement, qui doit
        comprendre obligatoirement un médecin.
\end{enumerate}

\paragraph{}
De façon à vérifier que ces conditions sont validées, un audit est demandé par
l'ASIP\@. Lors du renouvellement, celui ci est externe et à la charge de
l'hébergeur, autrement il est effectué par l'ASIP dans cadre de la première
demande.

            \subsubsection{Accès aux données}

\paragraph{}
Dans le cas de l'hébergement d'une application permettant au patient d'accéder
directement à des informations liées à son dossier médical, certaines
contraintes de sécurité sont requises, détaillées dans l'article L1110--4 du
code de la santé publique.

\paragraph{}
Lors de la demande d'agrément, l'hébergeur doit renseigner la façon dont sont
intégrées l'identification et l'authentification du patient dans son
application.


\paragraph{Identification\\}
L'hébergeur doit assurer que les méthodes d'identification utilisées par
l'application attribuent le bon identifiant au bon patient, en ne présentant
aucun cas de doublons et de risques de collisions entre dossiers de patients
différents.


\paragraph{Authentification\\}
Afin de préserver la sécurité des accès, un moyen d'authentification forte doit
être impérativement utilisé. Plusieurs solutions sont proposées.

\paragraph{}
L'hébergeur peut utiliser un couple d'identifiant/mot de passe associé à un
mot de passe à usage unique appelé OTP\@. Il doit dans ce cas indiquer la
procédure d'envoi du mot de passe au patient, la façon de l'initialiser à la
première connexion ainsi que la procédure d'envoi des informations liées à
OTP\@.

\paragraph{}
Il peut également utiliser un certificat électronique contenu dans une carte à
puce, mais comme pour OTP, la façon de délivrer au client ce certificat doit
être détaillée, ainsi que les moyens par lesquels il est protégé.

\paragraph{}
Pour l'identification comme pour l'authentification, quand l'hébergeur n'est
pas directement en lien avec le patient mais joue le rôle de prestataire, il
doit définir de manière explicite les principes que s'engage à respecter son
client afin de garantir la bonne sécurité des données.


        \subsubsection{Hébergeurs agréés}

\paragraph{}
Une association d'hébergeurs agréés pour la conservation de données médicales a
été créée en 2010 pour faciliter les nouveaux arrivants à effectuer les
démarches nécessaires. Elle se nomme l'AFHADS\footnote{\emph{AFHADS~: }
Association Française des Hébergeurs Agréés de Données de Santé à Caractère
Personnel} et permet de jouer le rôle de lobby pour les sociétés membres. Début
2015, lors des votes sur la loi relative au renseignement, elle s'est manifesté
pour clarifier la situation sur les données médicales~: elles sont censées être
gardées confidentielles, mais le gouvernement souhaitait avoir un regard sur
celles-ci. Suite à cela, des arrangements non publics avec quelques-uns des
plus gros membres ont été signés.

\paragraph{}
Comme exemples de sociétés agréées pour l'hébergement de données médicales
peuvent être cités SFR et Orange, pour leurs offres d'hébergement ``Cloud
Santé'', CEV, AZNetwork ou encore Pro BTP, première société donc le domaine
d'activité principal ne se situe pas l'informatique. Il est rassurant de voir
que la liste des adhérents ne contient aucune société dont le secteur
d'activité est la publicité, bien que SFR et Orange possèdent des divisions qui
y sont liées.

        \subsection{Champs d'application}

\paragraph{}
Pour stocker des données médicales, il faut tout d'abord obtenir l'autorisation
du patient concerné~: aucune donnée de ce type ne peut être conservée sans
autorisation préalable.

\paragraph{}
Les données ne peuvent pas être vendues à une société tierce, et ne peuvent
être utilisées par exemple par des sociétés publicitaires. La partie de
l'activité portant sur l'hébergement de ces données doit être nettement séparée
du reste, comme demandé pour l'obtention de l'agrément, de façon à éviter toute
ambigüité. Cette restriction se veut rassurante avec l'arrivée, par exemple, de
Google dans le domaine médical avec Google X, dont la source de revenus
principale est la publicité.

\paragraph{}
La situation la plus courante pour un hébergeur de données médicales est un
contrat signé avec un professionnel de la santé. L'hébergeur ne doit pas
obligatoirement posséder une expertise dans le domaine médical, tant que les
conditions du contrat respectent celles imposées par l'agrément. Arrivé en fin
de contrat, l'hébergeur se doit de remettre à son client l'intégralité des
données le concernant ainsi que ses patients, tout en prouvant que les données
ont bien été effacées.


        \subsection{La carte vitale}

\paragraph{}
La carte Vitale a été mise en service dès 1998, elle permet d'identifier de
manière unique un assuré social, et est un composant du système
d'information SESAM-Vitale. En plus de l'identifiant unique, la carte
contient les informations suivantes~:

\begin{itemize}
    \item l'identité du titulaire et de ses ayant-droits de moins de 16 ans
    \item le numéro de sécurité sociale du titulaire
    \item le régime d'assurance maladie et l'organisme de rattachement
    \item les droits à la couverture maladie universelle complémentaire (CMUC)
    \item les droits à l'exonération du ticket modérateur (frais médicaux normalement à la charge du patient)
\end{itemize}

\paragraph{}
Une nouvelle version, appelée Vitale 2 a quant à elle été introduite en
2007 pour remplacer progressivement les cartes existantes.  À la différence
de la première version, une photo de l'assuré est présente sur la face et
des informations supplémentaires sont présentes sur sa mémoire.

\begin{itemize}
    \item le médecin traitant
    \item la mutuelle
    \item la personne à prévenir en cas d'accident
    \item les informations sur l'accès aux soins dans l'UE
    \item la gestion des accidents du travail
    \item la carte de donneur d'organe du titulaire
\end{itemize}

\paragraph{}
Techniquement, La carte vitale est équipée d'un microprocesseur lui
permettant de chiffrer et déchiffrer ses données de manière autonome. En
plus des mesures de chiffrement le protocole permettant de questionner la
carte est maintenu secret. Ainsi, pour pouvoir lire ou écrire une carte
vitale, il est nécessaire d'utiliser une bibliothèque propriétaire, détenue
par la caisse d'assurance maladie.
