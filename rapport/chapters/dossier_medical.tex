    \section{Le dossier médical}


\paragraph{}
Le dossier médical patient désigne l'ensemble des informations recueillies par
les professionnels du domaine médical à propos de l'état de santé d'un patient.
Ces données, stockées sous forme de numérique ou sur papier sont indispensables
pour assurer le suivi d'un patient. Bien que dans la plupart des établissements
médicaux, les dossiers sont stockés sous forme numérique le partage des données
n'est pas systématique avec d'autres établissements.

\paragraph{}
Bien au contraire, la CNIL veille au respect de la vie privée des patients,
notamment en définissant des normes de sécurité et des règles d'accès très
strictes. Pour autant, l'accès au données est quasi-systématiquement possible
pour le patient, et à d'autres personnes dans des cas restreints, très encadrés
par la loi.


        \subsection{Le contenu du dossier médical}

\paragraph{}
Les informations contenues dans le dossier médical sont initialement prévues
pour documenter l'état de santé d'un patient et suivre ses traitements lors de
son admission dans un établissement de santé.  Un professionnel de la santé
travaillant avec le patient a accès à ce dossier pour déposer les données
formalisées.

\paragraph{}
Le terme ``formalisé'' n'est pas défini, mais la loi du 4 mars 2002 prévoit que
``les informations formalisées accessibles au patient doivent s'entendre comme
présentant un certain degré d'élaboration et de validation''. Par ailleurs,
d'après une décision de la Cour d'Appel de Paris, les lames d'examen biologique
ne sont pas des informations
formalisées\footnote{\url{http://documentation.fhp.fr/documents/caparis_20080213.pdf}}

\paragraph{}
Les informations que le dossier médical est susceptible de contenir sont et
réparties en trois catégories~:

\begin{itemize}
    \item Les informations recueillies par l'établissement lors de
        consultations.

        Ces informations comprennent le motif d'admission dans l'établissement, les
        soins administrés, les résultats d'examens médicaux.

        Cette catégorie comprend également les correspondances entre les
        professionnels de la santé.

    \item Les informations établies en fin de séjour d'un patient dans un
        établissement.

        Cette catégorie comprend le compte rendu d'hospitalisation et la lettre
        rédigée à la sortie du patient, les prescriptions et ordonnances.

    \item Les informations recueillies auprès de personnes tierces, ou à propos
        de personnes tierces.

        De telles données peuvent être pertinentes dans un dossier médicales, mais
        doivent être consignées à part préserver la vie privée des tiers en
        question.
\end{itemize}

\paragraph{}
Le patient possède un droit d'accès et de rectification au dossier, comme prévu
par la loi informatique et libertés, à l'exception du contenu de la troisième
catégorie, susceptible de contenir des informations concernant une personne
tierce. Un exemple des données susceptibles de figurer dans la troisième
catégorie est l'analyse par un psychologue du comportement des proches d'un
patient~: les informations sont pertinentes dans le cadre du traitement du
patient, mais la vie privée des proches est une raison suffisante pour ne pas
divulguer l'information au patient.


        \subsection{L'accès aux données}

\paragraph{}
Lorsqu'un patient demande l'accès à son dossier, l'établissement médical est dans l'obligation
de lui fournir l'intégralité des données des deux premières catégories.

\paragraph{}
Dans certains cas, il est possible que l'accès au dossier soit proposé au patient sous conditions~:

\begin{itemize}
    \item Si le patient est hospitalisé en psychiatrie, la consultation du
        dossier peut être refusée, ou doit être faite en présence d'un médecin.

    \item Pour les patients mineurs, l'accès au dossier médical peut être donné
        aux parents, mais le patient mineur peut demander que l'accès aux
        informations se fasse par l'intermédiaire d'un médecin, et dans ce cas,
        s'opposer à ce que le médecin communique les informations.

    \item Dans le cas d'une personne décédée, les successeurs peuvent accéder
        au dossier médical complet ou en partie. Pour obtenir des informations,
        un successeur doit présenter un motif à la consultation du dossier. Les
        motifs valides de consultation sont~:
            \subitem{\textbullet\,} Prendre connaissance de la cause du décès.
            \subitem{\textbullet\,} Défendre la mémoire du défunt.
            \subitem{\textbullet\,} Faire valoir ses droit.
\end{itemize}

\paragraph{}
Le troisième motif résulte d'une décision du Conseil d'État par rapport à un
contentieux\footnote{\url{http://www.legifrance.gouv.fr/affichJuriAdmin.do?oldAction=rechJuriAdmin&idTexte=CETATEXT000008133767&fastReqId=583477576&fastPos=1}}~:
l'établissement refuse de communiquer le dossier médical d'une personne
décédée, sous prétexte que l'ayant droit dont émane la demande est en litige
avec un autre ayant droit, et que les informations son susceptibles d'être
utilisées dans le cadre du litige.  Le refus de communication du dossier dans
ce cas est alors jugé illégal.

Dans certains cas, les établissement échouent à fournir les dossier médicaux
complets ou bien dans les délais impartis par la loi. Par exemple, une décision
de la Cour d'Appel de Marseille établit que la communication partielle du
dossier médical d'un patient décédé, provoquant la confusion sur la cause du
décès constitue un préjudice moral à l'encontre des ayant droit du
patient.\footnote{\url{http://documentation.fhp.fr/documents/caamarseille_20080313.pdf}}
