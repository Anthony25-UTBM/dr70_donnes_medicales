    \section*{Conclusion}
\addcontentsline{toc}{section}{Conclusion}

\paragraph{}
En ayant anticipé un stockage massif des informations médicales, la CNIL et le
ministère de la santé ont réussi à prévenir les possibles fuites de données que
peut induire cette activité, en encadrant les hébergeurs autant sur la sécurité
que sur la confidentialité des données.

\paragraph{}
Malgré les règlementations contraignantes sur les données médicales, les
technologies utilisées aujourd'hui par les professionnels de la santé
parviennent à concilier efficacité et sécurité. Un témoin de ce succès est la
carte Vitale, qui parvient, en tant qu'élément de la vie quotidienne ainsi que
composant d'un système d'information d'envergure nationale, de transmettre les
informations efficacement.

\paragraph{}
Un des points complexe de la modernisation des systèmes d'information du monde
médical est la délégation des responsabilités du stockage à des acteurs privés,
et la diffusion aux personnes autorisées via internet. Pour autant, bien que
potentiellement existantes, nous n'avons réussi à trouver de jurisprudence
portant sur la fuite de données médicales~: ce résultat ne prouve pas pour
autant l'efficacité complète des mesures établies --- car ce type d'évènement
n'est pas forcément ébruité --- mais se veut rassurant quant à cette confiance
accordées aux hébergeurs privés.
